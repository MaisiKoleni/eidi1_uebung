% !TEX TS-program = pdflatex
% !TEX encoding = UTF-8 Unicode

% This is a simple template for a LaTeX document using the "article" class.
% See "book", "report", "letter" for other types of document.

\documentclass[11pt]{exam} % use larger type; default would be 10pt

\usepackage[utf8]{inputenc} % set input encoding (not needed with XeLaTeX)

%%% Examples of Article customizations
% These packages are optional, depending whether you want the features they provide.
% See the LaTeX Companion or other references for full information.

%%% PAGE DIMENSIONS
\usepackage{geometry} % to change the page dimensions
\geometry{a4paper} % or letterpaper (US) or a5paper or....
\geometry{margin=1in} % for example, change the margins to 2 inches all round
% \geometry{landscape} % set up the page for landscape
%   read geometry.pdf for detailed page layout information

\usepackage{graphicx} % support the \includegraphics command and options

\usepackage[parfill]{parskip} % Activate to begin paragraphs with an empty line rather than an indent

%%% PACKAGES
\usepackage{booktabs} % for much better looking tables
\usepackage{array} % for better arrays (eg matrices) in maths
\usepackage{paralist} % very flexible & customisable lists (eg. enumerate/itemize, etc.)
\usepackage{verbatim} % adds environment for commenting out blocks of text & for better verbatim
\usepackage{subfig} % make it possible to include more than one captioned figure/table in a single float
% These packages are all incorporated in the memoir class to one degree or another...

\usepackage{hyperref}
\usepackage{listings}
\usepackage{xcolor}
\usepackage{color}

\usepackage{amssymb}
\usepackage{textcomp}

\usepackage{minted}
\usepackage[T1]{fontenc}
\usepackage{lmodern}

%%% SECTION TITLE APPEARANCE
\usepackage{sectsty}
\allsectionsfont{\sffamily\mdseries\upshape} % (See the fntguide.pdf for font help)
% (This matches ConTeXt defaults)

%%% ToC (table of contents) APPEARANCE
\usepackage[nottoc,notlof,notlot]{tocbibind} % Put the bibliography in the ToC
\usepackage[titles,subfigure]{tocloft} % Alter the style of the Table of Contents
\renewcommand{\cftsecfont}{\rmfamily\mdseries\upshape}
\renewcommand{\cftsecpagefont}{\rmfamily\mdseries\upshape} % No bold!

%%% END Article customizations

%%% The "real" document content comes below...
\title{Übungsaufgaben EIDI 2 \\ \small \color{magenta}Version 0.1.0}
\author{Christian Femers}
%\date{} % Activate to display a given date or no date (if empty),
         % otherwise the current date is printed 


\definecolor{jcom}{rgb}{0.5,0.3,0.3} 
\definecolor{jnum}{rgb}{0.3,0.3,0.9}
\definecolor{jstring}{rgb}{0.1,0.5,0.2}
\definecolor{jkeyw}{rgb}{0.7,0,0.3}

\lstdefinestyle{mystyle}{
    commentstyle=\color{jcom},
    keywordstyle=\color{jkeyw},
    numberstyle=\tiny\color{jnum},
    stringstyle=\color{jstring},
    basicstyle=\ttfamily\small,
    breakatwhitespace=false,         
    breaklines=true,                 
    captionpos=b,                    
    keepspaces=true,                 
    numbers=left,                    
    numbersep=8pt,                  
    showspaces=false,                
    showstringspaces=false,
    showtabs=false,                  
    tabsize=2,
	frame=single
}
\lstset{style=mystyle}

\newcommand{\code}[1]{\mintinline{Java}|#1|}

\usepackage[german]{babel}
\usepackage{csquotes}

\pointsinrightmargin
\bracketedpoints
\pointpoints{Pinguin}{Pinguine}
\checkboxchar{$\square$}

\begin{document}
\maketitle

\begin{questions}
\question Was sind \textbf{keine} Java-Schlüsselwörter?
\begin{checkboxes}
\choice \texttt{final}
\CorrectChoice \texttt{const}
\CorrectChoice \texttt{var}
\choice \texttt{short}
\CorrectChoice \texttt{\_}
\choice \texttt{case}
\choice \texttt{class}
\CorrectChoice \texttt{main}
\choice \texttt{static}
\choice \texttt{thow}
\choice \texttt{throws}
\CorrectChoice \texttt{null}
\end{checkboxes}
\question Zu was evaluieren die folgenden Java-Ausdrücke? 
\begin{parts}
\part \code{3 - 0}: \hrulefill % 3
\part \code{-7 / 2}: \hrulefill % -3
\part \code{-7 / 2.0}: \hrulefill % -3.5
\part \code{-7d / 2}: \hrulefill % -3.5
\part \code{true ? 3 : 2 + 1}: \hrulefill % 1
\part \code{0.25 * 8}: \hrulefill % 2.0
\part \code{42 * 2 + "Niugnip" + 42 + 2}: \hrulefill % 84Niugnip422
\part \code{(long) 1.0 + "java" + 1}: \hrulefill % 1java1
\part \code{4 + ~-1 >= 5 ? 2 * 0 : 3 / 2 + 2}: \hrulefill % 3
\part \code{'a' + 25 - 'z'}: \hrulefill % 0
\part \code{(8745 / 61) + (83 / 0)}: \hrulefill % java.lang.ArithmeticException: / by zero
\part \code{0 / 1 == 0 ? null : "hallo"}: \hrulefill % null
\part \code{(byte) 127 + 1}: \hrulefill % 128
\end{parts}
\question Betrachten Sie den folgenden Code-Auszug:
\begin{minted}[frame=single,tabsize=4,linenos]{Java}
Integer input = getUserInput();

if (input == (Integer) 42)
	System.out.println("Antwort gefunden");
else
	System.out.println("Weitersuchen");
	
if ("83".equals("" + (int) input))
	System.out.println("83 gefunden");
\end{minted}
Welche der folgenden Aussagen treffen zu? Nehmen Sie an, dass der Code kompiliert und betrachten sie ihn als Algorithmus. Gehen Sie nur von dem aus, was sie sehen können. 
\begin{checkboxes}
\choice In Zeile 3 wird der Wert von \texttt{input} mit \texttt{42} verglichen.
\CorrectChoice In Zeile 3 werden Objekte auf Referenzgleichheit geprüft
\CorrectChoice \texttt{Antwort gefunden} wird möglicherweise für eine Eingabe von \texttt{42} ausgegeben.
\choice Für die Eingabe 42 wird nie \texttt{Antwort gefunden} ausgegeben werden.
\CorrectChoice Möglicherweise \texttt{Weitersuchen} für eine Eingabe von \texttt{42} ausgegeben.
\choice Möglicherweise wird \texttt{Antwort gefunden} und \texttt{Weitersuchen} für eine Eingabe von \texttt{42} ausgegeben.
\CorrectChoice Bei der Eingabe von \texttt{83} wird immer \texttt{Weitersuchen} und \texttt{83 gefunden} ausgegeben.
\choice \texttt{83 gefunden} wird nie ausgegeben werden.
\choice In Zeile 8 wird auf Referenzgleichheit geprüft.
\CorrectChoice In Zeile 8 werden zwei Strings zeichenweise miteinander verglichen.
\CorrectChoice Möglicherweise \texttt{Weitersuchen} für eine Eingabe von \texttt{42} ausgegeben.
\choice Bei Zeile 3 wird nie eine \texttt{NullPointerException} geworfen werden.
\CorrectChoice Bei Zeile 8 wirft \texttt{equals} eine \texttt{IllegalArgumentException}.
\CorrectChoice Zeile 8 wirft möglicherweise eine \texttt{NullPointerException}.
\choice Die Ausgabe ist nicht deterministisch.
\CorrectChoice Möglicherweise wird gar nichts in die Konsole ausgegeben.
\end{checkboxes}
\question Zu welchem Wert evaluieren die folgenden Ausdrücke, vorausgesetzt die \code{int}-Variable \texttt{x} hat vor jeder Teilaufgabe den Wert 0?
\begin{parts}
\part \code{x++}: \hrulefill % 0
\part \code{x = x = x++}: \hrulefill % 0
\part \code{x++ + ++x}: \hrulefill % 2
\part \code{--x != ++x}: \hrulefill % true
\part \code{--x - --x - x}: \hrulefill % 3
\part \code{x++ == x++ ? x-- - 1 : --x + 1}: \hrulefill % 2
\part \code{x++ * x++ * x++}: \hrulefill % 0
\end{parts}
\question Was sind erlaubte Bezeichner für Variablen in Java (ab Version 9)?
\begin{checkboxes}
\choice \texttt{fin al}
\CorrectChoice \texttt{const}
\CorrectChoice \texttt{var}
\CorrectChoice \texttt{µPC}
\CorrectChoice \texttt{\$name}
\choice \texttt{\_}
\CorrectChoice \texttt{\_\_}
\choice \texttt{public}
\CorrectChoice \texttt{CLASS}
\choice \texttt{42sinn}
\CorrectChoice \texttt{main}
\CorrectChoice \texttt{\_mäin\_}
\choice \texttt{pingu!n}
\choice \texttt{null}
\end{checkboxes}
\question Erstellen sie das Kontrollflussdiagramm zu folgendem Java-Code-Auszug. Die Methode \code{read()} gibt dabei einen \code{int} zurück, \code{write(int)} gibt den übergebenen \code{int}-Wert auf der Konsole aus.
\begin{minted}[frame=single,tabsize=4,linenos]{Java}
final int a = read();
int b = a;
OUTER: for(int i = 0; i < 5; i++) {
	switch(b) {
		case 1:
		case 2: b++; continue;
		case 5:
		case 4: b--; break;
		case 3: break OUTER;
	}
	write(b);
}
write(b);
\end{minted}

\end{questions}

\end{document}